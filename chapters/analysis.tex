%!TEX root = ../main.tex

\section{Analysis} % (fold)
\label{sec:analysis}

In this chapter we will look into the advantages and disadvantages of sonification. Some of the advantages to consider are mentioned in the previous chapter, and so the advantages in this chapter will mainly be regarding using sound to convey data. We will also look into different contradicting terms in relation to sonification. Lastly, we look further into the sonification techniques that are available to us. This is so we can delimit what techniques we are to use when presenting data during our test. The advantages and disadvantages are so we know what to look out for in this regard.


\FloatBarrier
\subsection{Advantages and Disadvantages} % (fold)
\label{sub:advantages_and_disadvantages_of_sonification}

There are several advantages for sonification. Not only can it represent data and learn patterns and cycles in said data, but it can also aid the disabled and assist workers in different fields with sound patterns and alarms (Section~\ref{sec:preanalysis}).

However, as with everything there are certain disadvantages of sonification. The disadvantages can all be lead from the same disadvantages that sound has. Sound can be subjective to the listener. Sound is also incredibly hard to describe in certain cases, you cannot mimic sound or make others interpret data without them being able to hear it. Listening to sound representing data can require learning or inherent talent, and this learning can take a longer time than looking at visually represented data, although that depends on the type of data. The type of data represented also needs to be taken into account for whenever a sound or pattern is chosen. 


\enquote{It is hard to represent a spatial arrangement with sound. You need to use the appropriate display.}~\cite*[pp.21-22]{Feder2012}

To give a more concrete example of an advantage, sonification was used to study a model of an artificial heart pump, where sound was used to indicate the different metrics of the heart. It was reported that it seemed easier to determine things such as the frequency of blood cells, and status of the heart valve~\cite*[pp.24]{Barrass1999}.
This is in contrast to the standard method of looking at change in color on a monitor. 

Another example is sonification used to indicate seismic events caused by the fracturing of rock walls during a three months dig. Engineers had a visualization to indicate the fractures, but the events happened too quickly for them to properly percieve. To solve this problem they added a short sound that had varying volume, depending on the magnitude of the event~\cite*[pp.24]{Barrass1999}.

From these two examples, it is clear that when dealing with data that has a variety of events or events happening over longer periods of time, using sound can make the data easier to comprehend.


With examples of advantages and disadvantages in the representation of data through sonification, and how it is used, it is also valid to research existing types of sonification that represent weather data. This will give indications towards pre existing solutions to sonification of weather data, and might enlighten plausible methods of how we should develop a prototype.


\subsubsection{Weather data represented using sound} % (fold)
\label{ssub:weather_data_represented_using_sound}

Our focus is weather data, but we also need to understand how to use auditory displays. In cases where the interpretant needs to keep his eyes focused on other visual data, it is possible to use sound to receive auditory data at the same time. For example, data was recorded and sonified from the recording of an actual turbine, to simulate a turbines flow~\cite*{Kramer1993}.

Another example of this is from the CSIRO water treatment plant, where they had a large amount of data to be perceived at once. To simulate the amount of downpour, they used sonification to generate rain-like sound~\cite*[pp.24-25]{Barrass1999}.
The result of this experiment was that a worker was able to learn the sounds through repetition, and could make distinction on how much downpour would fall in a year~\cite*[pp.24-25]{Barrass1999}.

What we can gather from this is that it is possible for a user to perceive information through an auditory display, while also focusing on other things. However, it also tells us that in some cases the user needs to be conditioned or taught what the different sonifications mean before they are able to fully utilize a display.

% subsubsection weather_data_represented_using_sound (end)

% subsection advantages_and_disadvantages_of_sonification (end)

\FloatBarrier
\subsection{Analogic vs Symbolic representation} % (fold)
\label{sub:analogic_vs_symbolic_representation}

When designing a product or anything that uses sonification, there have been some methods specified. An article called, Using Sonification, written by Gregory Kramer states:

\enquote{An analogic representation is one in which there is an immediate and intrinsic correspondence between the sort of structure being represented and the representation medium.}~\cite*[pp.26]{Barrass1999}

Some examples have been given in this article in the form of a Geiger counter and an auditory thermometer~\cite*[pp.26]{Barrass1999}.
Kramer also gives examples representing symbolic sonification, and these include computer beeps and car sounds~\cite*[pp.26]{Barrass1999}.

What we can derive from this is that analogic display usually means something physical with added audio to assist the user, where symbolic means giving the sounds themselves symbols and meanings.

% subsection analogic_vs_symbolic_representation (end)

\FloatBarrier
\subsection{Semiosis} % (fold)
\label{sub:semiosis}


\enquote{Semiotics is the study of signs and their creation and interpretation.}~\cite*{Chandler2007}

What this means is that the process of semiosis is making something that has no meaning on it’s own, and giving it a meaning based on what we think it should be. This process usually involves a “signifier” and a “signified"~\cite*{Chandler2007}.
A signifier would typically be a sound pattern, and a signified carries the meaning that the signifier gives. The signifier and the signified are non-material objects and only have the value or meaning that we weigh them with~\cite*{Chandler2007}.

In other words, semiotics is when we give symbols some other meaning, possibly using sound to convey whatever it is we wish to represent. 

% subsection semiosis (end)

\FloatBarrier
\subsection{Sonification usage} % (fold)
\label{sub:sonification_usage}

Sonification can be split up in 4 different kinds of function types, when going by the sonification theory of Bruce N. Walker and Michael A. Ness in their article, Theory of Sonification.~\cite*[pp.3]{walker2011}. 

The four different types are as follows:

\begin{enumerate}
    \item Alarms, alerts, and warnings
    \item Status, process, and monitoring messages
    \item Data exploration
    \item Art and entertainment
\end{enumerate}


\paragraph{Alerting functions} % (fold)
\label{par:alerting_functions}

We all know the basic function of an alert. A loud noise plays repeatedly indication that something has happened. An alarm can serve as a warning or a reminder. There is for instance a different intention of a fire alarm and an alarm clock. Although alerts function as a means of getting the attention of bystanders, it does not say much about the event which has occurred:


\enquote{The message conveyed is information-poor. For example, a beep is often used to indicate that the cooking time on a microwave oven has expired. There is generally little information as to the details of the event—the microwave beep merely indicates that the time has expired, not necessarily that the food is fully cooked.}~\cite*[pp.4]{walker2011}. 

% paragraph alerting_functions (end)


\paragraph{Status and progress indicating functions} % (fold)
\label{par:status_and_progress_functions}


Status and progress functions have a purpose similar to that of alerting functions, but to understand the data correctly, a visual display or another form of continuous data is required to fully understand the situation at hand~\cite*[pp.5]{walker2011}. 

It requires the interpretant to pay attention to small changes in the auditory feedback. This relates to the example given (Section~\ref{ssub:examples_of_sonification}) that we have previously discussed, where surgeons focus their eyes on the patient while listening to the e.g. heart rate monitors.


% paragraph status_and_progress_functions (end)


\paragraph{Data exploration functions} % (fold)
\label{par:data_exploration_functions}

When speaking of sonification, this function is usually what is referred to. Data exploration is meant to inform a listener using auditory displays. Like in the examples given in [Section~\ref{sub:advantages_and_disadvantages_of_sonification}], data exploration is the function of transferring some, or complete sets of data through sonification~\cite*[pp.5]{walker2011}. 
This is different from other forms of sonification where one might only capture the present state of an event.

% paragraph data_exploration_functions (end)


\paragraph{Art and Entertainment} % (fold)
\label{par:art_and_entertainment}

This form of sonification is not necessarily meant to deliver information to the listener. However, data and information can still be used to produce sound, but typically there’ll also be musical elements involved. Despite the possible involvement of instruments, as with the other types of sonification there must be an involvement of computers to generate the sound from data~\cite*[pp.5]{walker2011}. 


The above covers the different sonification functions, and in which cases they can be useful to alert or notify interpretants.

We will generally focus on the first category of sonification, but not in the sense that we need to alarm people.
We know that sonification can notify you in many different ways. Some forms require a passive listener and some requires an active listener, maybe both. The approaches for sonification are also different depending on the task or the message you wish to convey. We also know that sonification can have interaction methods that require the user to physically interact with an object to generate sonification to work. According to our research regarding sonified weather data, it is stated that engineers needed to become familiar with the sound to know the intensity of the rain. We will therefore expect that if the users are hearing the sound for the first time, they may not give the correct answer.

For the prototype we will be using event based sonification (Section~\ref{ssub:sonification_techniques}). Continuous sonification could also be used, if we were to take weather data and directly transform it into sonification, but this will not be the case in the prototype. 

 The type we will be using is data exploration, since we wish to translate weather data into sounds that can be heard by the end user during testing.
Using event based sonification means we will employ parameter mapping. Theoretically this would mean that we have the ability to change the sound as our weather data changes, which seems to be the ideal choice.
So to summarize what we have learned and what we will use in our prototype, we will dive right into a list of requirements.

% paragraph art_and_entertainment (end)



% section analysis (end)
