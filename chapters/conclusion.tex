%!TEX root = ../main.tex

\section{Conclusion} % (fold)
\label{sec:conclusion}

Through the process of the project it became apparent that research areas such as sonification and weather data was obligatory to get an understanding of how it was possible, if possible, to convey weather information through sonification, as efficiently as visually presented weather data, which was deemed as the initial problem statement. 

The analysis of the research lead to an understanding of knowledge of advantages and disadvantages of sonification along with sonification usage and state of the art of weather data presented using sound. 
Based upon this research it was possible to create a list of requirements which should suffice in the construction of a design that would aid in the answering of the final problem statement.

The test performed were build upon giving us the feedback needed for a Kruskal-Wallis test. 
This was done to see if there could be detected any difference between a visual represented weather forecast and an audio represented one, which was our sonifications based upon prior research in the design phase. 
The test were split into three sub tests (Visual , Audio, and a fusion of both aspects) that would provide three data sets that could be compared later and detect any differences. 
The data also provided pie charts to show each of the weather condition data elements with the distribution of answers.

The answers provide knowledge as to which sonifications were intuitively understood, and the acquired responses made it possible to answer the final problem statement.

We believe It is possible to convey the weather information that we have chosen to present to the test subjects, solely as a nonspeech auditory display, using sonification techniques.
However, there is nothing in the test results that indicate that the sonifications are as intuitively understandable as the visually presented weather information. 
This is shown by the Kruskal-Wallis test, seeing as not all of the visual tests are similar to the results from the sonifications. 
Furthermore, the Kruskal-wallis test shows that the high value data was above the confidence interval of 0.05 and therefore shows that there is a difference between the visual representation and the sonification that represents the same data.

A statement, which has proven to be fitted on the obtained test results can be found in the pre-analysis (See Section ~\ref{sec:preanalysis},0); as Gregory Kramer writes: \enquote{The main problem is that it is difficult to provide adequate context for interpreting sonifications of data.}

While the sonifications are defined by ranges of values being low, medium and high it is possible to make the sonifications intuitive. 
It would however be questionable if the weather data should be sonified as continuous data values that would represent any specific weather condition, as we deem it impossible to make intuitive, but the sheer development would be possible. 
We can therefore conclude that, with the chosen weather data, it is possible to convey the weather information through sonification. 
The extend of successful implementation of the sonifications are however limited by the choice of delimiting values to low, medium or high. 
More specific data content for the sonifications, and the success of such implementation would require additional research.

% section conclusion (end)