%!TEX root = ../main.tex

\section{Evaluation} % (fold)
\label{sec:evaluation}

The evaluation will be divided into three separate test scenarios. 
As we are looking into how we can convey weather data through sonification, and how intuitively understood this information compared to visual representations of similar data is. 
It will be possible to evaluate upon by isolating and comparing data from several tests, where one test is of the produced sonifications, another test with visual elements and one test where both visuals and sonifications is tested.

More specifically described, a test will be conducted solely where the weather sonifications are presented to test subjects. 
Similarly, a test will be conducted where the visual representations will be presented with the same procedure as the sonification, and finally a test where both visual representations along with the sonification of the data.
This procedure will make it possible to compare what the test subjects can deduct from the presented data, and allow us to make comparisons towards what information the visual, sonification and combined representations of data is conveying, and thereby make it possible to evaluate upon how well the sonification formulate weather data.


\subsection{Method} % (fold)
\label{sub:method}

The primary reason of the evaluation is to establish knowledge to determine if the list of requirements has been fulfilled and to make it possible to gain satisfying results, which help answer the final problem statement.

The implemented prototype contains sonifications of specific weather categories, where these categories have been further categorised into sonifications of different values, which represent low, medium and high values of a specific weather category.

A Kruskal-Wallis test will be conducted on the gathered test results, which will be divided into three separate tests. 
This will ensure that it is possible to distinguish between the results gathered from the low values, medium values and the high values of the weather data, and will provide 3 separate box plots and associated data, which will allow us to prove or disprove the null hypothesis and to make assumptions and further analysis, which will help to answer the final problem statement.


\subsection{Procedure} % (fold)
\label{sub:procedure}

Here is the procedure for the test explained in detail. 
This chapter will describe the different aspects of the test such as sampling, observations, and guidelines for the testing. 
The same test procedure will be done on three separate presentations of weather conditions: visual presentation, audial representation, and combined presentation.
The general data for the tests are presented below.

Amount of participants: 20 subjects per category, per representation.
Test subject sampling: Convenience sampling: Aalborg university students.
Estimated length of test: 1 minute per subject.
Test location: Area on and around Aalborg university.
Date and time: When appropriate.


\subsubsection{Test setup} % (fold)
\label{ssub:test_setup}

The test will not require any specific location. 
For the testing only two people are present, i.e. 1 test subject and the test conductor. 
Only the combined test required two conductors for convenience purposes. 
The test subject and observer will be placed opposite each other during the test.
Other individuals, in this case other test subjects, will be placed out of view to avoid bias by previous subjects answers. 

% subsubsection test_setup (end)


\subsubsection*{Test Materials} % (fold)
\label{ssub:test_materials}
A listing of required materials for each test.

\paragraph{Visual test} % (fold)
\label{par:visual_test}

\begin{itemize}
    \item Visual samples: A set of images each representing a separate weather condition. The samples will be presented to the subject one at a time to avoid that the subject makes assumptions based on the other samples. 
    \item Notation sheets: Sheets of paper where the presented samples and respective responses are recorded.
\end{itemize}
% paragraph visual_test (end)

\paragraph{Sound test} % (fold)
\label{par:sound_test}

\begin{itemize}
    \item Audio samples: A set of audio snippets each representing a separate weather condition. The samples will be produced on scene using Pure Data and will not be prerecorded.
    \item Notation sheets: Sheets of paper where the presented samples and respective responses are recorded.
\end{itemize}
% paragraph sound_test (end)

\paragraph{Combined test} % (fold)
\label{par:combined_test}

\begin{itemize}
    \item Visual samples: The same samples from the visual test.
    \item Audio samples: the same samples from the audial test.
    \item Notation sheets.
\end{itemize}
% paragraph combined_test (end)

% subsubsection test_materials (end)


\subsubsection{Test Execution} % (fold)
\label{ssub:test_execution}

The testing procedure differs slightly in the three tests. 
In the visual test the subject is presented an image of a weather condition. 
The images do not contain the actual weather data they try to represent. 
During the audio test the subject is presented with an audio sample produced in Pure Data, and in the combined test the subject is presented with both an image and an audio sample of the same weather condition. 

The overall testing execution looks like this:

\begin{itemize}
    \item The subject is placed opposite the observer.
    \item The observer presents the subject with a sample.
    \item The presented sample is noted by the observer and the observer asks the subject if the sample presents a low, medium, or high value.
    \item The observer notes the answer and continues to the next sample until a sample from each category has been answered.
\end{itemize}
Before each test the subjects are informed that they are to be presented with 5 different samples (one for each category). 
They will be asked a question to each sample and that they are only to answer \enquote{high}, \enquote{medium}, or \enquote{low} to the questions. 
In order to ensure that each subject has the same understanding of the samples, they are given some context for each sample. 
This is done through the question asked to each sample. 
The question contains the weather condition but not the actual data i.e. \enquote{How much wind is in this image, high, medium, or low?} in case of a visual sample.
The subject will not be given any further clarification as we are looking for the intuitive guess and not the qualified guess.

% subsubsection test_execution (end)


\subsection{Hypothesis} % (fold)
\label{sub:hypothesis}

Based on the formulation of the final problem statement, the scientific experiment can be divided into two separate hypotheses. 
One hypothesis which specifies that there are no significant difference between the samples, and one hypothesis which specifies that there is some significant difference between the samples.


\subsubsection*{Null Hypothesis} % (fold)
\label{ssub:null_hypothesis}

There is no significant difference between the understanding of the non-speech auditory display of weather data using sonification techniques, compared to the understanding of visually presented weather information. 
% subsubsection null_hypothesis (end)

\subsubsection*{Alternative Hypothesis} % (fold)
\label{ssub:alternative_hypothesis}

There is some significant difference between the understanding of the non-speech auditory display of weather data using sonification techniques, compared to the understanding of visually presented weather information.

% subsubsection alternative_hypothesis (end)

\subsection{Test Delimitation} % (fold)
\label{sub:test_delimitation}

In order to prove or disprove the Null hypothesis, a statistical test is specified. 
The method of establishing what test will be used, the structure from Andy Field \& Graham Hole is utilized~\Cite[Ch. 8]{Field2003}.

\subsubsection{Type of data that is collected} % (fold)
\label{ssub:type_of_data_that_is_collected}

The data that is collected from the test participants is considered scores. 
As we are collecting data through a scientific study, the data that is contributed from the test participants are nominal data, being data that uses numbers to define categories or range of values which the test subjects think is correct.

What is obtained from the test participants is a value from 1-3 indicating low, medium or high values of a specific weather condition, which indicates their intuitive thought of what the formulated weather data represents.

% subsubsection type_of_data_that_is_collected (end)

\subsubsection{Independent variables} % (fold)
\label{ssub:independent_variables}

Independent variables, being something that is manipulated by us, is the weather conditions. 
There are one independent variable in this study, being the specific formulation of a single weather condition. 
As an example, the test subjects are presented with low/little rain as a visual image. 
Other test subjects are also presented with low/little rain, but as a sonification, or both. 
Then the test subjects provide scores, indicating what they actually think is presented.

% subsubsection independent_variables (end)

\subsubsection{Experiment design} % (fold)
\label{ssub:experiment_design}

As we are looking for differences or similarities between groups of people where we have altered independent variables, an experimental design is utilized.

% subsubsection experiment_design (end)

\subsubsection{Independent measures} % (fold)
\label{ssub:independent_measures}

Since we look for intuitively understood sonification of weather data, each participant will only be subjected to one condition of the test. 
This will allow us to obtain a single score from each participant in either the sonification, visual, or sonification \& visual test. The study can therefore be described as a wholly independent measure designed test.

The test subjects will be allowed to contribute with several answers within the same condition, but will not be allowed to contribute to others.

% subsubsection independent_measures (end)

\subsubsection{Non-parametric} % (fold)
\label{ssub:non_parametric}

As we have no pre-defined assumptions to the test, and to what extent there might be certain characteristics, the study can be described as non-parametric.

% subsubsection non_parametric (end)

\subsubsection{Test groups} % (fold)
\label{ssub:test_groups}

As specified as the independent variable, there are three different groups which will contribute with scores.

% subsubsection test_groups (end)

\subsubsection{Kruskal-Wallis Test} % (fold)
\label{ssub:kruskal_wallis_test}

By delimitating the above mentioned steps, it is defined that a Kruskal-Wallis test can be utilized to prove or disprove the Null hypothesis.


\enquote{A Kruskal-Wallis test compares between the medians of two or more samples, to determine if the samples have come from different populations.}~\cite{Gaten2000}
This will make it possible to know if the samples are similar or different, given the independent variable and can tell us if the test subjects understand visual, auditory or both in different or similar manners, and the actual distribution of data.


There are however, some limitations to the Kruskal-Wallis test.
If no significant difference in our data is found, the samples can not be concluded as to being similar.
If there are any significant differences, it is not possible to make specific assumptions as to what contributed to the significant difference. 
Further analysis and testing would be required. \cite*{Gaten2000}.
% subsubsection kruskal_wallis_test (end)

% subsubsection test_delimitation (end)

% subsection hypothesis (end)

% subsection procedure (end)

% subsection method (end)

\subsection{Results} % (fold)
\label{sub:results}

The results from the test are here presented and described. 
Discussions and interpretations of the results are conducted in the Evaluation: Discussion (Section~\ref{sub:discussion})

The gathered results can be found in appendix~\ref{sec:test_results}.

\begin{figure}[!htbp]
    \centering
    \includegraphics[width=1\textwidth]{images/Evaluation1.jpg}
    \caption{Test result snippet}
    \label{fig:evaluation1}
\end{figure}

As seen on figure~\ref{fig:evaluation1}, there are three categories. 
Low elements, medium elements, and high elements. 
This example only covers the visual weather data. 
The results gathered from Sonification, and sonification \& Visual elements can be found in the appendix~\ref{sec:test_results}.

The numbers indicates the responses from 20 test participants. 
These are labels of 1-2-3 which indicates low, medium and high values.

The rows indicates responses from a single test participant.

\begin{itemize}
    \item Low values has the label: 1
    \item Medium values has the label: 2
    \item High values has the label: 3
\end{itemize}

The \enquote{inverse} represents that the data from the specific data set has been inversed. 
As visibility had the correct answer of 3, but as the labelling indicates, the low value label is 1, therefore to make the responses match the other results, the data is inverted. 
This way, it is possible to know that each response of 1 is correct or opposite.


It can be deduced from figure~\ref{fig:evaluation1}, that if a low weather condition has been answered and labeled with 1, then the answer is correct, as the image illustrates a weather condition of low value. 
Likewise with medium values which has the answer 2 and high values with an answer of 3. 
If a medium sound is heard, and the test subject answers correctly with a response of medium, then 2 is noted.

\subsection{Kruskal-Wallis Test Results} % (fold)
\label{sub:kruskal_wallis_test_results}

The test results will be divided into three separate Kruskal-Wallis tests, in order to evaluate upon low data, medium data and high data separately, to make the distinction between the independent variable more clearly. 
The outcome is a Boxplot, respectively of low, medium and high weather data elements along with the data descriptions of the three Kruskal-Wallis tests.


\subsubsection*{Low values: Kruskal-Wallis ANOVA table} % (fold)
\label{ssub:low_values_kruskal_wallis_anova_table}

\begin{figure}[!htbp]
    \centering
    \includegraphics[width=.7\textwidth]{images/Evaluation2.jpg}
    \caption{ANOVA table low data}
    \label{fig:evaluation2}
\end{figure}

The P value(Prob>Chi-sq) is 0.0042 which is under 0.05. 
Therefore the null hypothesis is rejected and accepts the alternative hypothesis.

This means that we accept that there is some significant difference between the understanding of the non-speech auditory display of weather data using sonification techniques, compared to the understanding of visually presented weather information, specifically for low value data, which has been specified as the alternative hypothesis.

% subsubsection low_values_kruskal_wallis_anova_table (end)

\FloatBarrier
\subsubsection*{Low values: Kruskal-Wallis test results} % (fold)
\label{ssub:low_values_kruskal_wallis_test_results}

\begin{figure}[!htbp]
    \centering
    \includegraphics[width=.5\textwidth]{images/Evaluation3.jpg}
    \caption{Boxplot low data}
    \label{fig:evaluation3}
\end{figure}

Generally the boxplot indicates similar levels of medians but the visual scale has a different distribution than \enquote{Sound and Visual} and \enquote{Sounds}.

The \enquote{Visual} element is comparatively short, as the inner quartile range is overlapping, which suggests that a high number of responses are 1. 
A majority of participants answered correctly, but a low number of participants, indicated by the \enquote{plus}, that 2 and 3 was also answered by a lower population of the test participants.

% subsubsection low_values_kruskal_wallis_test_results (end)

\FloatBarrier
\subsubsection*{Medium values: Kruskal-Wallis ANOVA-Table} % (fold)
\label{ssub:medium_values_kruskal_wallis_anova_table}

\begin{figure}[!htbp]
    \centering
    \includegraphics[width=0.7\textwidth]{images/Evaluation5.jpg}
    \caption{ANOVA table medium data}
    \label{fig:evaluation5}
\end{figure}

As seen on figure \ref{fig:evaluation5} the P value (Prob>Chi-sq) is 0.0088 and is therefore below the confidence interval of 0.05 and does therefore disprove the Null hypothesis and accept the alternative hypothesis.

% subsubsection medium_values_kruskal_wallis_anova_table (end)

\FloatBarrier
\subsubsection*{Medium values: Kruskal-Wallis Boxplot results} % (fold)
\label{ssub:medium_values_kruskal_wallis_boxplot_results}

\begin{figure}[!htbp]
    \centering
    \includegraphics[width=.7\textwidth]{images/Evaluation7.jpg}
    \caption{Boxplot medium data}
    \label{fig:evaluation7}
\end{figure}

The Boxplot of the medium values, see figure \ref{fig:evaluation7}, illustrates the distribution of the test results amongst the acquired data. The median of the three data samples are generally similar, but the distribution of responses are somewhat different. 
The median amongst the samples are 2, which is also the label which represents middle value data.

The \enquote{Visual} boxplot indicates a minimum value which collides with the lower quartile and describes that atleast 50 percent of the test participants has answered 1 in the visual test of medium values. 
Another 25 percent has answered 2, and the maximum value indicates that 25 percent has answered 3, which is labelled as the high value.

The \enquote{Sound and Visual} illustrates, as the upper and lower quartiles, along with the maximum and minimum is overlapping, illustrates that a majority of the test participants has answered 2, being the correct answer. 
Only a few test participants has answered 3 which is the label of high values.

Lastly, the \enquote{Sound} being the sonification of the medium valued weather data indicates that the median is 2, with a upper quartile of 25 percent having answered 3, along with 50 percent having answered 1.

% subsubsection medium_values_kruskal_wallis_boxplot_results (end)

\FloatBarrier
\subsubsection*{High values: Kruskal-Wallis ANOVA-Table} % (fold)
\label{ssub:high_values_kruskal_wallis_anova_table}

\begin{figure}[!htbp]
    \centering
    \includegraphics[width=0.7\textwidth]{images/Evaluation6.jpg}
    \caption{ANOVA table high data}
    \label{fig:evaluation6}
\end{figure}

The P. value (Prob>Chi-sq) is 0.0561 and is above 0.05 then indicates that we fail to reject the null hypothesis with specifically high value weather condition results.

Therefore here is no significant difference between the understanding of the non-speech auditory display of weather data using sonification techniques, compared to the understanding of visually presented weather information specifically for high value weather conditions.

% subsubsection high_values_kruskal_wallis_anova_table (end)

\FloatBarrier
\subsubsection*{High values: Kruskal-Wallis Boxplot results} % (fold)
\label{ssub:high_values_kruskal_wallis_boxplot_results}

\begin{figure}[!htbp]
    \centering
    \includegraphics[width=.7\textwidth]{images/Evaluation8.jpg}
    \caption{Boxplot high data}
    \label{fig:evaluation8}
\end{figure}

The High-value boxplot of which the null hypothesis is accepted, there are no significant difference in the samples. 
They are however not similar.
The Visual sample indicates that 25\% of the test subjects thought of some or more of the visual interpretations of weather conditions as being a middle value. 
There are however in the sound and visual sample below 25\% who thought that is was a medium value. 
And with the sound sample, below 25\% of the test participants thought that the sonification indicated either low or medium, leaving 75\% test participants answering correctly.

% subsubsection high_values_kruskal_wallis_boxplot_results (end)

% subsubsection kruskal_wallis_test_results (end)

\FloatBarrier
\subsubsection*{Graphical test results} % (fold)
\label{ssub:graphical_test_results}

As the Kruskal-Wallis test only provides knowledge of statistical differences of the data groups, it is also important to gain knowledge of the specific weather condition elements which proved to be successful or unsuccessful. 
In order to elaborate upon what sounds, compared to the visual elements had any/if any specific impact. 
Therefore a graphical structure of the results are developed (See appendix~\ref{sub:test_result_graphs}).

\begin{figure}[!htbp]
    \centering
    \includegraphics[width=.7\textwidth]{images/Evaluation4.jpg}
    \caption{Snippet of graphic chart: Test Results}
    \label{fig:evaluation4}
\end{figure}

As seen on figure~\ref{fig:evaluation4}, a snippet of the pie charts are presented. 
On the row A1-3, the three temperature results is comparable. 
The pie charts each represents 20 responses and indicates the participants answers of the visual elements, sound and visual elements and the sound element. 

Each pie chart illustrates the number of answers for low, medium and high which is illustrated by a percentage of the chart in a color and a number that indicates the amount of test participants who answered that specific value. 
The title refers to the sound that has been played to the test subjects.

% subsubsection graphical_test_results (end)

% subsection results (end)

\FloatBarrier
\subsection{Discussion} % (fold)
\label{sub:discussion}

Here, the discussion and interpretations of the results are presented. 
What is deducted from the data and what thoughts as towards what the answers indicate will be described and explained.

\subsubsection{Kruskal-Wallis tests Boxplots} % (fold)
\label{ssub:kruskal_wallis_tests_boxplots}

\paragraph{Low Value:} % (fold)
\label{par:low_value_}
\hspace{0pt} \\
The low value boxplot (Figure~\ref{fig:evaluation3_2}) compliments the ANOVA table and illustrates significant difference of the independent variables between the \enquote{visual}, the \enquote{sonification} and \enquote{both combined}, which is the altered independent variable.

We can see that a large amount of participants answered correctly on the visual test, but a few who answered two or three. 
Although the numbers are very few, it is not enough to increase the upper quartile, meaning that the ones who answered incorrectly are below 25\%.  
With the \enquote{sound} and \enquote{sound \& visual}, the upper quartile indicates that from the median up to maximum, 50\% of test participants answered incorrectly, as the correct answer was 1, being the label of low. 
A large amount of the test participants answered 2 (25\%) or 3 (25\%). 
This indicates that a large amount of the test participants misinterpreted the sonification of low weather conditions.

\begin{figure}[!htbp]
    \centering
    \includegraphics[width=.5\textwidth]{images/Evaluation3.jpg}
    \caption{Boxplot low data}
    \label{fig:evaluation3_2}
\end{figure}

% paragraph low_value_ (end)

\paragraph{Middle Value:} % (fold)
\label{par:middle_value_}
\hspace{0pt} \\
The middle value Boxplot (See figure \ref{fig:evaluation7_2}) indicates a difference in the \enquote{visual}, \enquote{sound} and \enquote{sound \& visual} test results, which also is indicated as the null hypothesis is accepted, and that we therefore conclude there there is a some significant difference between the understanding of the non-speech auditory display of weather data using sonification techniques, compared to the understanding of visually presented weather information.

On all three groups, the medians are 2, which is the correct label for the middle value, and illustrates that a majority of the test participants answered correctly in each of the three tests.

The notched boxplot \enquote{Visual} illustrates that 50\% of the test participants answered 1, which is incorrect and lower than the intended formulation. 
There are however below 25\% who answered incorrectly in the test where Sound \& Visual was presented, and only 3 (high value) was answered wrong by a small number of the test participants.

Lastly the boxplot for \enquote{Sound} indicate that a majority answered correctly with 2, but 25\% answered 3, and another percentage of test subjects answered 1, which indicates that there are a majority of test subjects that misinterpreted the middle value sonifications. 

\begin{figure}[!htbp]
    \centering
    \includegraphics[width=.7\textwidth]{images/Evaluation7.jpg}
    \caption{Boxplot medium data}
    \label{fig:evaluation7_2}
\end{figure}

% paragraph middle_value_ (end)

\paragraph{High Value:} % (fold)
\label{par:high_value_}
\hspace{0pt} \\
The High-value boxplot of which the null hypothesis is accepted, there are no significant difference in the samples. 
They are however not similar.
The Visual sample indicates that 25\% of the test subjects thought of some or more of the visual interpretations of weather conditions as being a middle value. 
There are however in the sound and visual sample below 25\% who thought that is was a medium value. 
And with the sound sample, below 25\% of the test participants thought that the sonification indicated either low or medium, leaving 75\% test participants answering correctly.

\begin{figure}[!htbp]
    \centering
    \includegraphics[width=.7\textwidth]{images/Evaluation8.jpg}
    \caption{Boxplot high data}
    \label{fig:evaluation8_2}
\end{figure}

% paragraph high_value_ (end)

% subsubsection kruskal_wallis_tests_boxplots (end)

\FloatBarrier
\subsection{Pie charts} % (fold)
\label{sub:pie_charts}
Although the boxplot and ANOVA tables indicates similarities or differences between the samples, they do not describe what specific sounds/visuals had any impact or what could have been done differently or what specific sounds were actually successful in conveyance of sonifications of weather data.

The following appendix~\ref{sub:test_result_graphs} indicates the numbers of correct/incorrect responses of each weather category, and makes it possible for comparisons.

Though it is possible with our test results to elaborate upon which sonifications were successfully implemented, and to which degree, it is not possible to make assumptions based upon acquired evidence what actually was a success, and what was incorrect in the implementation of the sonifications.

More specifically, it implies that we are able to observe the number of correct or incorrect responses from the test subjects, but we are unable to answer why we obtain these specific data sets based upon the acquired scientific test results.

As we are only able to make assumptions based upon the acquired data, this will be discussed in the following chapter (See discussion, section~\ref{sec:discussion}).

% subsection pie_charts (end)

% section evaluation (end)