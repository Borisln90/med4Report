%!TEX root = ../main.tex

\section{Discussion} % (fold)
\label{sec:discussion}

In this chapter the project will be discussed and reflected upon. 
This includes insight as to what could have been done differently to improve the acquired results, or and what impact any decision throughout the report might have had. 
Ultimately the discussion will lead to how the problem statement can be answered i.e. how well the problem was solved, and what this result implies.

For the visual part of our test, (See chapter XX, section XX) we conducted the test using images created by ourselves, based upon the visual pre test(See chapter xx, section xx), which could have changed/affected the test results. We chose to create our own pictures since we believed it would not matter if they were taken from a website or drawn by us. We based our drawings on information given from the visual pre-test (see Design chapter). The test results indicates that not everyone could see what the pictures represented(Ref til appendix pie charts. XX) This is of course an issue and for further testing these images should perhaps be replaced with something not based on our pre-test, but pictures found on weather represented websites or authored by a professional. Pictures as these may be more recognizable for people since there’s a chance they have seen them in weather forecasts on TV or in newspapers, which could in turn lead to different test results.


Something that might have made a difference to our test results is the implemented filters. The idea about changing some aspects in the sounds might not have worked out as well as we had hoped.From our observations during testing, we observed that even though we changed some of the sound it still sounded the same for the testers. An example of this that our rain sound might sound similar at the medium and high values. This is most likely because they never heard the other sounds for the two other values, so they had nothing to mentally compare with. It was also not possible to make it sound like there was close to no rain, without making it sound unnatural. We believe that it could be more efficient to have different sounds instead of changing one sound to represent the different weather data values.


In chapter XX, section XX it was delimitated that the range of weather condition data should not be sonified in a continuous manner, where every value had a sonification, but rather in sections which were represented by low, medium, or high values in which the specific weather condition was implied. 
By making the decision of delimitating the amount of sonifications, and altering the sonifications from exact values to an indication of a specific range of values formulated by a label of low, medium, or high where the understanding is individual to each test subject, the sonification itself is altered to formulate another type of content, rather than the actual weather data. 

What this implies is that we make a predetermined decision towards the possibility of making an exact sonification of the entire ranges of values in a specific weather condition, and to be intuitively understandable, which we believe to be impossible.

The obtained test results have both positive and negative indications regarding the success of intuitively understanding the visual and sonified weather data, but the data only represents low, medium, and high values, where a continuous sonifications of weather data should indicate degrees, wind speed etc. where decimals are included, if it was to be an exact interpretation of the designated set of data. 

In chapter XX, section XX, an example is provided where a similar procedure was used. In this example it is stated that engineers had to repeatedly listen to the different predefined values for a while, before being familiar enough with the sound to understand what it meant. This is unlike the intuitive sonifications which a user should be able to understand without having to learn what data the sonification implies.

The evaluation chapter in this report provides data from the test subjects that indicate differences or similarities in the data, when using the Kruskal-Wallis tests, and provides detailed information about specific weather condition data, which was correctly or incorrectly answered. See section (XX) for evaluation results. However, there was not obtained test data, which elaborates upon why any of the presented weather data conditions were answered correctly or incorrectly, but we do have theories.


The pie charts (see appendix XX) show the responses for each weather data information, indicated by a single pie chart in the appendix. The data does not specifically reflect the following , but we assume that weather conditions which actually have a sound are more likely to be intuitively understood than weather data which have no sound (but this is understood via associations). An example is Wind speed (See appendix XX, K32). When there is wind, it is possible to hear the wind, unlike temperature. There is no sound of temperature, but associations that makes one think that there must be a specific temperature because of associated sounds. 100 percent of test participants answered correctly on high wind so a plausible solution can be that the sound is intuitive as there already exists specific sounds for wind. However, as indicated by low rain (Appendix xx,  B6), where there is a sound for low amount of rain, 17 out of 20 answered incorrectly even though rain ought to be intuitive much like wind speed. The reason for this error could however be our choice of sound representation which could indicate some other values. e.g. mid or high value of the data, leading the test subjects to answer incorrectly.

It can also be argued that associations to weather data is an individual interpretation, which makes it unlikely to make accurate sonifications of weather data, as there is no correct choice of sonifications to illustrate an association to the specific weather data, but only a plausible majority who may think that a specific sound/sonification accurately conveys that data.

While interpreting the results of the test, it became apparent that while it could be possible to evaluate upon each specific visualisation and sonification of the weather data, it would not result in any new information, which would allow for a more thorough and accurate answer to the final problem statement. This is because of the delimitation that has been made early in the design phase. The delimitation is the choice of only conducting sonifications of low, medium and high values. A re-design could be conducted where a questionnaire would be added to the predefined test structure where test subjects would be asked to describe what they think could be altered within the visualizations and sounds/sofications. This would give indications as to what could be altered in order to gain more accurate results, which would be reflected in the pie charts of each specific set of data values.(See appendix XX)

However, even if the test would be conducted and the sonifications and visualizations would be modified numerous times until the data from the tests would indicate that 100 percent of the test subjects answered correctly, the sonications are still delimited to low, medium and high values. 

For this very reason it would seem unimportant to conduct further testing and re-design of the product. We would gain evidence to support the claim that it is possible to intuitively convey weather data, if the data is limited to low, medium and high values, but the test results is already indicating such assumptions to some extend whereof a number of test participants are answering correctly on each of the sonifications.


With what we have discussed so far, we can begin to discuss answering the final problem statement.


"To what extent is it possible to convey weather information, solely as a nonspeech auditory display, using sonification techniques, and be as intuitively understandable as visually presented weather information, where intuitive is defined as knowing by intuition?"


“To what extent...” is as mentioned defined by the choice of delimiting the continuous weather data values to labels of low, medium and high. We can therefore argue that it is indeed possible to convey weather information, solely as a nonspeech auditory display, using sonifications techniques, where the conveyed data is weather conditions with low, medium and high values. 

The pretests (see chapter XX section XX)  indicates specific visualisations and sounds which could be utilized to the sonification of the data, which supports that the conveyed data is intuitive, meaning that a majority of the test participants could relate and understand the specific weather condition along with the specified data value, being low, medium or high. Also, as the results indicate, only a limited number of test participants answered correctly, but we assume that a refined product would make it possible to gain more satisfactory results, obtaining results that indicate a high margin of correctly answered questions within the low, medium and high labeled values of each weather condition.

Lastly, as we have also delimited the weather information (See chapter xx, section xx), we can only account for the specified weather conditions as mentioned.

To elaborate upon the possibility of making sonifications to represent continuous values of a weather condition, in addition to the labels of low, medium and high, would require further testing, but we deem it possible.However, we question the usability, as the difficulty of conveying such small differences and changes would be very high. It would be very difficult for a person to either understand intuitively, remember, or even learn it in the first place.


% section discussion (end)