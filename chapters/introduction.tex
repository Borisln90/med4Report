%!TEX root = ../main.tex

\section{Introduction} % (fold)
\label{sec:introduction}

The weather forecast is a tricky thing. 
Everyone likes to talk about them, everyone has their own opinion about what is important and their own preferences on how the weather data should be presented to them. 
One common way of getting the weather forecast is through mobile applications. 
They all present the weather data in different detail, some of them show a lot of data at the same time using raw data in tables spanning a lot of pages. 
This could be complicated and difficult for many to read. 
Other weather applications show only a short set of data in a single page, using symbols and images to illustrate the most important weather conditions, such as illustrations of an umbrella when it is going to rain.
These applications might oversimplify the message and the user might end up making the wrong decisions. 
Some of these applications will be investigated further in our analysis of the problem.

Common for all of them is that they only present the data visually and not audibly. 
Even on the TV, where audio is present, the weather forecast is primarily visual in form of graphs, radar images and numbers, where a very trustworthy person in a nice suit interprets it for you.

Sonification, which is the use of non-speech audio to convey information, is also a potential way of bringing data about the weather conditions. 
Over the course of this report we will explore the areas of sonification and test if it is possible to use sound to create a weather forecast and still use this inborn intuition to understand the message.
We will also test if it is possible to combine visual and sound to get a better understanding of the message of the forecast.

% section introduction (end)


