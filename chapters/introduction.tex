%!TEX root = ../main.tex

\section{Introduction} % (fold)
\label{sec:introduction}

The weather forecast is a tricky thing. 
Everyone likes to talk about them, everyone have their own opinion about what is important and preferences on how the weather data should be presented to them. 
Some people like for the forecasts to be very detailed, while others prefers them over simplified. 
This is why there are so many weather apps available on the market today and each app presents the forecasts just a bit different from the others. 
Some apps presents the forecasts with a lot of details combined with pretty pictures and animations, while others only presents the most important data and a simplistic animation.

Common for all of them is that they only presents the data visually and not audibly. 
Even on the Tv the focus is visual in form of graphs, radar images and numbers, which a very trustable person in a nice suit interpretes for you.

Presented on their own, weather data do not necessarily make a lot of sense to the individual. 
Unless the individual are aware of, what a wind speed of a meter per second feels like and its effect on other weather conditions. 
However the human being posses an intuition which tells what 30*C feels like and based on that, we make our choice on weather to bring a sweater or not for the day.

Over the course of this report we will explore the areas of sonification and test if it is possible to use sound to create a weather forecast and still use this inborn intuition to understand the message.
We will also test if it is possible to combine visual and sound to get a better understanding of the message of the forecast.

% section introduction (end)


